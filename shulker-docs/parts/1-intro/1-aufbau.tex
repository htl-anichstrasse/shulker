\chapter{Aufbau}

\section{Komponenten}

Das Projekt besteht aus den folgenden \textbf{drei} Komponenten.


\begin{itemize}
    \item \textbf{Shulker-Connect}: Shulker-Connect ist ein auf dem Raspberry-Pi laufender Webserver, der als Schnittstelle für die Kommunikation zum Türschloss dient. 
    \item \textbf{Shulker-Mobile}: Shulker-Mobile ist eine Smartphone-App, mit der Nutzer des Türschlosses dieses verwalten können.
    \item \textbf{Shulker-Core}: Shulker-Core ist eine auf dem Raspberry-Pi laufende Software, die die Benutzeroberfläche des Touchscreens und das elektrische Schloss selbst steuert.
\end{itemize}

Die genaue Funktionsweise der einzelnen Komponenten wird in den folgenden Teilen der Diplomschrift noch genauer beschrieben.

\section{Testumgebung}

Für Test und Demonstrationszwecke wurde das Türschloss in einer miniatur-Tür eingebaut. 
