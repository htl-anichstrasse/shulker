\chapter{Allgemeines}
Das Vorzeigemodell dient zur Veranschaulichung des gesamten Projekts. Es ist nicht als tatsächliche
praktische Implementation gedacht, sondern als eine kleine Visualisierung. Natürlich dient es auch als Beweis dafür,
dass das gesamte Shulker-System voll funktionsfähig und in der Realität anwendbar ist.

Das Modell wird nach Fertigstellung des Diplomprojekts in die HTL-Anichstraße gebracht und dort vorgestellt.

\section{Planung}
In die Planung des Vorzeigemodells flossen verschiedene Faktoren ein. Die wichtigsten Entscheidungsgrundlagen lauten:

\begin{itemize}
    \item Das Modell muss irgendwo gelagert und physisch in die HTL-Anichstraße gebracht werden. Um die Umständlichkeit
    dieses Vorgehens zu minimieren, wurde die Größe des Türrahmens in etwas kleineren Dimensionen realisiert.

    \item Das Modell dient nur als Veranschaulichung und Beweis der Funktionalität von Shulker. Daher müssen keine besonders aufwendige
    Sicherheitsmaßnahmen getroffen werden.
\end{itemize}