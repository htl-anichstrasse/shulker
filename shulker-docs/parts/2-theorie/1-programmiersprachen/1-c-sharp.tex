\section{C-Sharp}
\label{C-Sharp}

\textit{C-Sharp} ist eine typsichere objektorientierte Allzweck-Programmiersprache.
In C-Sharp geschriebene Anwendungen können in \textit{.NET-Core} ausgeführt werden. 
Da der C-Sharp Syntax ähnlich zu anderen Sprachen in der C-Sprachfamilie ist, werden die meisten
Programmierer keine großen Schwierigkeiten haben, C-Sharp Programmcode zu schreiben bzw. zu lesen.  
\cite{csharpmicrosoft}

\subsection{.NET-Architektur}
C-Sharp Programme werden durch \textit{.NET-Core} ausgeführt. 
Als erstes werden C-Sharp Programme in einen Zwischensprache kompiliert.

\subsection{Warum C-Sharp?}
Wir haben uns für C-Sharp entschieden, da C-Sharp über das \textit{ASP.NET Core} Web-Framework eine elegante Lösung bietet, dynamische Webservices zu erstellen.

Die Integrierung von C-Sharp in \textit{.NET-Core} ermöglicht eine Plattformübergreifende Entwicklung, welche aufgrund des am Raspberry-PI laufendem ARM-Basierenden RaspiOS benötigt wird.