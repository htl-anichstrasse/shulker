\section{C-Sharp}
\label{C-Sharp}

\textit{C-Sharp} ist eine typsichere objektorientierte Allzweck-Programmiersprache.
In C-Sharp geschriebene Anwendungen können mittels \textit{.NET}, \textit{CLR} und bestimmten 
\textit{Klassenbibliotheken} ausgeführt werden. 
Da der C-Sharp Syntax ähnlich zu anderen Sprachen in der C-Sprachfamilie ist, werden die meisten
Programmierer keine großen Schwierigkeiten haben, C-Sharp Programmcode zu schreiben bzw. zu lesen.  
\cite{csharpmicrosoft}

<<<<<<< HEAD
=======
\subsection{.NET-Architektur}
C-Sharp Programme werden durch \textit{.NET-Core} ausgeführt. 
Der in C-Sharp geschriebene Programmcode wird 

>>>>>>> 8afeec1f28b4e2f7653360cdec7fe311bea44be6
\subsection{Warum C-Sharp?}
Wir haben uns für C-Sharp entschieden, da C-Sharp über das \textit{ASP.NET Core} Web-Framework eine elegante 
Lösung bietet, dynamische Webservices zu erstellen.

Die Integrierung von C-Sharp in \textit{.NET-Core} ermöglicht eine Plattformübergreifende Entwicklung, 
welche aufgrund des am Raspberry-PI laufendem ARM-Basierenden RaspiOS benötigt wird.

\subsection{.NET-Architektur}
Das .NET-Framework (bis 2020 .NET-Core) ist eine freie und offene Entwicklungs-Plattform. Mit .NET können eine breite Menge an
Applikationen geschrieben werden, diese können Plattformunabhängig auf diversen Betriebssystemen
und Architekturen ausgeführt werden. \cite{dotnetmicrosoft} 

\subsubsection{Programmiersprachen}
.NET unterstützt drei Programmiersprachen:
\begin{itemize}
    \item \textbf{C\#}: C-Sharp ist eine einfache und moderne Allzweck-Programmiersprache.
    \item \textbf{F\#}: F-Sharp ist eine schnell zu schreibende und performante Programmiersprache.
    \item \textbf{Visual Basic}: Visual Basic ist eine Programmiersprache mit einfacherem Syntax, für die keine
    neuen Funktionen entwickelt werden.  
\end{itemize}

\subsection{Common Language Runtime}
Die virtuelle Laufzeitumgebung von .NET heißt \textbf{Common Language Runtime} (CLR), diese wird von 
Microsoft bereitgestellt und führt den Programmcode aus.
CLR ist eine direkte Implementierung Microsofts der \textit{Common Language Infrastructure} (CLI).
Die Common Language Infrastructure ist ein internationaler Standard, der sowohl eine 
Programmiersprachen- als auch Plattformneutrale Entwicklung und Ausführung von Anwendungen ermöglicht.  

\subsection{ASP.NET Core}
\textit{ASP.NET Core} ist ein offenes Plattformunabhängiges Web-Framework, das es einem erlaubt,
schnelle und moderne Webanwendungen zu schreiben. ASP.NET Core wird in .NET ausgeführt.

ASP.NET Core ist eine Neugestaltung von ASP.NET, diese Neugestaltung ermöglichte ein 
schlankeres und modulareres Framework. 