\section{Dart}
\label{dart}

\textit{Dart} ist eine objektorientierte höhere Programmiersprache, die eine produktive Entwicklung für
mehrere Zielplattformen bietet. In Dart geschriebener Programmcode lässt sich für Web, Mobile und 
Desktop Zielsysteme kompilieren. 
Diese Kompilierung erfolgt entweder in Nativen-Code, oder für Web-Anwendungen eine Übersetzung in Javascript.
Der Syntax von Dart ist mit anderen Programmiersprachen in der C-Sprachfamilie vergleichbar.
Dart wird von Google entwickelt.
\cite{dartwikipedia}

\subsection{Flutter}
Dart bildet die Basis für das UI-Toolkit Flutter. Hierbei stellt Dart die Laufzeit für Flutter Applikationen dar.
Mit Flutter können Mobile-Applikationen (apps) entwickelt werden. 
Einen großen Vorteil, den Flutter für die Entwicklung von Mobilen-Applikationen bietet, ist, 
dass nur eine einzige Code-Basis für die Entwicklung von sowohl IOS und Android, als auch Web-Applikationen benötigt wird.
Drüber hinaus können Flutter Applikationen auch für Linux, Windows und MacOS kompiliert werden. 
Auch Flutter wird von Google entwickelt.
\cite{flutterwikipedia}

\subsection{Architektur des Flutter-Frameworks}
Die wichtigsten Komponenten des Flutter-Frameworks ergeben sich aus folgenden Bausteinen:
\begin{itemize}
    \item Dart-Plattform: Die Programmiersprache Dart bildet die Basis für das Flutter-Framework
    \item Flutter-Engine: Die Flutter-Engine ist eine portable Laufzeitumgebung für das Ausführen von Flutter-Applikationen
    \item Foundation-Library: Dieser Komponente stellt grundlegende Klassen und Funktionen, die für 
    Flutter-Applikationen benötigt werden, dar.
    \item Design-specific-widgets: Diese Designspezifische Widgets definieren das grundlegende Aussehen
    bestimmter grafischer Elemente des Zielsystems (z.B.: IOS, Android).
\end{itemize}

\subsection{Widgets in Flutter}