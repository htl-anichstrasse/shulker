\section{Flutter}
\label{flutter}
\textit{Flutter} ist ein UI-Toolkit von Google um Mobile-Applikationen (Apps) zu entwickeln.
Wir haben Flutter für die Entwicklung von Shulker-Mobile, der App zur Verwaltung des Türschlosses, verwendet.
Einen großen Vorteil, den Flutter für die Entwicklung von Mobilen-Applikationen bietet, ist, 
dass nur eine einzige Code-Basis für die Entwicklung von sowohl IOS und Android, als auch Web-Applikationen benötigt wird.
Drüber hinaus können Flutter Applikationen auch für Linux, Windows und MacOS kompiliert werden. 
\cite{flutterwikipediaEN}

\subsection{Dart}
\textit{Dart} ist eine objektorientierte höhere Programmiersprache, die eine produktive Entwicklung für
mehrere Zielplattformen bietet. Das Flutter-Framework wurde für diese Programmiersprache entwickelt.
Hierbei stellt Dart die Laufzeit für Flutter Applikationen bereit.
In Dart geschriebener Programmcode lässt sich für Web, Mobile und Desktop Zielsysteme kompilieren. 
Diese Kompilierung erfolgt entweder in Nativen-Code, oder für Web-Anwendungen eine Übersetzung in Javascript.

Der Syntax von Dart ist mit anderen Programmiersprachen in der C-Sprachfamilie vergleichbar.
Auch Dart wird von Google entwickelt.
\cite{dartwikipedia}

\subsection{Architektur des Flutter-Frameworks}
Die wichtigsten Komponenten des Flutter-Frameworks bilden folgende Bausteine: \cite{flutterwikipediaEN}
\begin{itemize}
    \item Dart-Plattform: Die Programmiersprache Dart bildet die Basis für das Flutter-Framework
    \item Flutter-Engine: Die Flutter-Engine ist eine portable Laufzeitumgebung für das Ausführen von Flutter-Applikationen
    \item Foundation-Library: Dieser Komponente stellt grundlegende Klassen und Funktionen, die für 
    Flutter-Applikationen benötigt werden, dar.
    \item Design-specific-widgets: Diese Designspezifische Widgets definieren das grundlegende Aussehen
    bestimmter grafischer Elemente des Zielsystems (z.B.: IOS, Android).
\end{itemize}

\subsection{Widgets in Flutter}
Der Grundbaustein für Flutter-Applikationen sind \textit{Widgets}.
Widgets sind Komponente in einem grafischen Anzeigesystem. 
In Flutter beinhalten ein Widget die Darstellung, Logik und Interaktion, die dieses Widget in der Applikation einnimmt.


Flutter stellt eine große Anzahl an vorgefertigten Widgets, die oft benötigt und im Kontext von Grafischen
Benutzeroberflächen bekannt sind, bereit. Diese Widgets können von Entwicklern beliebig verwendet werden.
Widgets in Flutter folgen den Designrichtlinien der jeweiligen Plattform (z.B.: Material Design für Android), 
definieren die Darstellung dieser allerdings selbst.

Aus Widgets lassen sich wiederum eigene, benutzerdefinierte Widgets erstellen.
Diese Benutzerdefinierten Widgets setzten sich aus beliebigen anderen Widgets zusammen 
und ermöglichen eine Wiederverwendung von Programmcode. \cite{flutterwikipediaDE}

\subsection{Arten von Flutter Widgets}
Grundsätzlich lassen sich Flutter Widgets in drei Arten unterteilen:
 \begin{itemize}
    \item Stateless widgets
    \item Stateful widgets
    \item Inherited widgets
 \end{itemize}

\subsubsection{Stateless Widgets}
Stateless widgets sind statische Widgets die nur bei ihrer Erstellung aktualisiert werden.
Das bedeutet, sie können nicht während der Laufzeit aktualisiert oder geändert werden. \cite{flutterstatelesswidgets}
Beispiele hierfür sind:
\begin{itemize}
    \item Text-Widget: Einfacher Text der auf dem Bildschirm angezeigt wird.
    \item Spacer-Widget: Das Spacer-Widget erstellt eine leere Fläche im UI,
    um einen Abstand zwischen zwei Widgets zu setzten.
\end{itemize}


Hier ein Beispiel mit gekürztem Programmcode für eine Ladeseite:
\begin{lstlisting}
class LoadingScreen extends StatelessWidget {
    @override
    Widget build(BuildContext context) {
        return Scaffold(
        child: Column(
            children: [
              Text("App ladet..."),
              CircularProgressIndicator(),
            ],
        )
        );
    }
}
\end{lstlisting}

Die erste Zeile des Programmcodes definiert eine neue Klasse namens \textit{LoadingScreen}.
Diese Klasse erbt von \textit{StatelessWidget} um ein Stateless-Widget zu definieren.
Die build Methode, mit dem Rückgabewert \text{Widget}, wird unter anderem bei der erstmaligen
Darstellung des Widgets aufgerufen. \cite{flutterstatelesswidgets}
Diese build Methode gibt ein \textit{Scaffold-Widget} zurück.
Ein Scaffold ist eine Implementierung der grundlegenden visuellen Layout-Struktur. 
Innerhalb dieses Scaffolds wurde ein body definiert, diesen zeigt das Scaffold-Widget am Bildschirm an.
In diesem body befindet sich ein \textit{Column-Widget}, dieses ordnet seine Unterobjekte vertikal an.
Letztlich befindet sich darin ein einfaches \textit{Text-Widget} und ein \textit{CircularProgressIndicator-Widget}, 
dieses Widget ist ein sich drehender Ladekreis. 

\subsubsection{Stateful Widget}
Stateful Widgets sind Flutter-Widgets, die einen veränderbaren Zustand besitzen.
Ein veränderbarer Zustand ist eine Information, die bei der Erstellung eines Widgets
gelesen wird, und sich während der Existenz des Widgets ändern kann.


Eine Änderung des Zustandes muss in Flutter mittels 
\begin{lstlisting}
    State.setState()
\end{lstlisting} gekennzeichnet werden. Wird mittels setState eine Variable des Zustandes geändert,
wird ein neuer Aufbau des dazugehörigen Objektes geplant. So kann z.B.: der Text eines Stateful-Widgets 
dynamisch geändert werden. Deshalb werden Stateful-Widgets immer dann verwendet, wenn sich Widgets während 
ihrer Existenz dynamisch ändern sollen. 

\subsection{Verwendete Flutter-Pakete}
Für Flutter entwickelte Dart-Pakete werden auf dem Package-Repository \textit{pub.dev} bereitgestellt.
Für die Entwicklung von Shulker-Mobile haben wir einige Pakete verwendet, um die Entwicklung der App zu
erleichtern.

Um ein Paket in ein Projekt einzubinden, muss eine Referenz in Form des Namen und der Version 
in die \textit{pubspec.yaml}-Datei eingefügt werden.


\subsubsection{Dio}
Das \textit{Dio}-Paket erlaubt es einem, Http-Anfragen von der Mobilen-Applikation aus zu versenden.
Wir haben es in Shulker-Mobile verwendet, um die Kommunikation mit dem Shulker-Connect API-Server zu ermöglichen

\subsubsection{qr\_code\_scanner}
Das \textit{qr\_code\_scanner}-Paket ermöglicht der App eine Kamera zu öffnen, in der 
QR-Codes gescannt und ausgelesen werden können. Wir haben dieses Paket verwendet, um den
Verbindungs-Prozess zum Türschloss zu erleichtern. Indem auf dem Touchscreen des Schlosses
ein QR-Code angezeigt wird, welcher die lokale IPv4-Adresse des dazugehörigen Raspberry-PI's
beinhaltet, kann dieser in der App gescannt werden, um eine manuelle Eingabe dieser Verbindungs-Parameter
zu vermeiden, und den Kopplungs-Prozess mit dem Türschloss für den Nutzer zu erleichtern. 

\subsubsection{shared\_preferences}
Das \textit{shared\_preferences}-Paket ermöglicht die Platform-Spezifische dauerhafte Speicherung
von Daten auf dem Mobilen-Endgeräten der Nutzer. In Shulker-Mobile kommt es zum Einsatz, um die 
Verbindungs-Parameter, also die lokale IPv4 des Raspberry-PI's des Türschlosses, zu speichern.

\subsubsection{check\_vpn\_connection}
Das \textit{check\_vpn\_connection}-Paket haben wir in Shulker-Mobile verwendet, um zu überprüfen, ob 
das Smartphone eine aktive VPN-Verbindung hergestellt hat. Dies zu wissen ist wichtig, da für die Verbindung zum Türschloss 
ein VPN-Tunnel in das Lokale Heimnetzwerk, in dem das Türschloss steht, benötigt wird. Diese VPN-Verbindung ermöglicht eine
sichere und verschlüsselte Verbindung, und erlaubt überhaupt erst die Netzwerk-Kommunikation zu dem sich in einem lokalen-Netzwerk
befindendem Türschloss.

\subsubsection{open\_settings}
Das \textit{open\_settings}-Paket verwenden wir, um dem Nutzer eine Abkürzung zur Einstellungs-Seite zu bieten, 
wo die derzeitige VPN-Verbindung verwaltet werden kann. Diese Abkürzung wird dem Nutzer dann angeboten, falls 
keine Verbindung zum Türschloss hergestellt werden konnte (Das bedeutet, das Smartphone befindet sich nicht im gleichen
Netzwerk wie das Türschloss), und keine aktive VPN-Verbindung herrscht. So kann sich der Nutzer komfortabel beim Start
der App mittels 2 Klicks mit dem VPN-Server des Routers im lokalen Netzwerk des Raspberry-PI's verbinden. 

\subsubsection{flutter\_localizations}
Das \textit{flutter\_localizations}-Paket wird in Shulker-Mobile verwendet, um in der \textit{main.dart} Datei
die Spracheinstellung der gesamten Applikation auf Deutsch zu setzen.

\subsubsection{cupertino\_icons}
Dieses Paket wird verwendet, um Cupertino-Icons (also IOS-Icons) in die Applikation einzubinden.

\subsubsection{uuid}
Das \textit{uuid}-Paket haben wir in das Projekt eingebunden, um UUID's der 4. Version zu erstellen.
Dies wird benötigt, um für neue Pin-Codes Universell eindeutige Bezeichnungen zu generieren.