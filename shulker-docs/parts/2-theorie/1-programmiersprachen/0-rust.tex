\section{Rust}
\label{Rust}

\textit{Rust} ist eine kompilierte Programmiersprache, welche auf mehrereren
Programmierparadigmen aufbaut. Die Sprache krönt sich damit, dass
sie extrem schnelle und sichere Binärprogramme erzeugt. Der gesamte Quellcode
vom gesamten Rust-Ökosystem ist quelloffen. Rust besitzt keinen Garbage-Collector,
Arbeitsspeicher wird mithilfe des Ownership-Models verwaltet.

\subsection{Warum Rust?}
Wir haben uns beim Komponent \textit{Shulker-Core}, also bei der Programmierung
des Touchdisplays und der Hardwareschaltung, sowie der Speicherung
von Daten, für Rust entschieden. Diese Entscheidung trafen wir primär
aus sechs Gründen:

\begin{itemize}
    \item Das Ownership-Model und die Einfachkeit von nebenläufiger Programmierung in Rust zwingt uns, vor allem im Bereich vom Touchdisplay, zu einer korrekten Lösung und weniger Fehlern
    \item Vorkenntnisse in \textit{Rust} und dem UI-Toolkit \textit{Slint} erleichtern uns das Arbeiten
    \item Die hohe Performance des Endprodukts minimiert die Hardwareauslastung
    \item Die Einfachkeit der Kompilierung macht es dem Nutzer leichter
    \item Da Rust eine Systemprogrammiersprache ist, ermöglicht es hardwarenahe Programmierung trotz hoher Abstraktionen
\end{itemize}

\subsection{Geschichte von Rust}
Rust enstand im Jahr 2006 als Hobbyprojekt von Graydon Hoare, einem Angestellten
bei Mozilla. Im Jahr 2009 begann Mozilla das Projekt zu unterstützen. Die erste
offiziell stabile Version (Version 1.0) wurde im Jahr 2015, also ganze neun Jahre danach, veröffentlicht.
Im Jahr 2020 wurden 250 Angestellte von Mozilla entlassen.  Ein großer
Teil des Rust Teams wurde damit entlassen. Das schien eine Bedrohung für
die Programmiersprache sein. Glücklicherweise wurde im darauffolgenden Jahr die
\textit{Rust Foundation} von den Firmen \textit{AWS, Google, Huawei, Microsoft und Mozilla} gegründet.
Rust wurde in der \textit{Stackoverflow Developer Survey}, einer großen Umfrage
für Programmierer, 2016, 2017, 2018, 2019, 2020 und 2021 zur am meisten geliebten Programmiersprache
ernannt. Ob das im Jahr 2022 so der Fall bleibt, wird sich zeigen.


\subsection{Programmierparadigmen}
Rust bietet verschiedenste Möglichkeiten, seinen Code zu gestalten. Unter
anderem borgt sich die Sprache Ideen von der funktionalen und objektorientierten
Programmierung. Auch das nebenläufige Programmieren, unter Rust-Usern als
\textit{Fearless Concurrency} bekannt, ist möglich. Man programmiert in Rust
primär imperativ, deklaratives Programmieren ist trotzdem auch ein wichtiger
Teil der Programmiersprache.

\subsection{Ownership-Model}
Das Ownership-Model ist Rusts größtes Verkaufsargument. Das Ownership-Model baut 
auf bestimmten Regeln auf, die den Arbeitsspeicher verwalten.
