\section{Rust}
\label{Rust}

\textit{Rust} ist eine kompilierte Programmiersprache, welche auf mehrereren
Programmierparadigmen aufbaut. Die Sprache krönt sich damit, dass
sie extrem schnelle und sichere Binärprogramme erzeugt. Der gesamte Quellcode
vom gesamten Rust-Ökosystem ist quelloffen.

\subsection{Warum Rust?}
Wir haben uns beim Komponent \textit{Shulker-Core}, also bei der Programmierung
des Touchdisplays und der Hardwareschaltung, sowie der Speicherung
von Daten, für Rust entschieden. Diese Entscheidung trafen wir primär
aus vier Gründen:

\begin{itemize}
    \item Das Ownership-Model und die Einfachkeit von nebenläufiger Programmierung in Rust zwingt uns, vor allem im Bereich vom Touchdisplay, zu einer korrekten Lösung und weniger Fehlern
    \item Vorkenntnisse in \textit{Rust} und dem UI-Toolkit \textit{Slint} erleichtern uns das Arbeiten
    \item Die hohe Performance des Endprodukts minimiert die Hardwareauslastung
    \item Die Einfachkeit der Kompilierung macht es dem Nutzer leichter
\end{itemize}

\subsection{Programmierparadigmen}
Rust bietet verschiedenste Möglichkeiten, seinen Code zu gestalten. Unter
anderem borgt sich Rust Ideen von der funktionalen und objektorientierten
Programmierung.