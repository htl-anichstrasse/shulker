\lstset{language=rust}
\chapter{Überblick}
\section{Allgemeines}
Shulker-Core besteht aus einer \textit{Rust-Binary}. Das bedeutet, dass der gesamte Quellcode von Shulker-Core zu einem einzigen
Binärprogramm kompiliert wird. Shulker-Cores gesamter Code befindet sich in einem Unterordner in der Shulker Git-Repository.

Wie der Name schon sagt, ist \textit{Shulker-Core} der Kern von Shulker. Das Binärprogramm ist dafür verantwortlich,
Pins zu speichern, Pins abzugleichen, das Touchdisplay zu steuern, das elektronische Schloss zu steuern und eine Schnittstelle
für \textit{Shulker-Connect} bereitzustellen. Um all diese Dinge effizient und sicher bewältigen zu können, wurde Rust als
Programmiersprache gewählt. Shulker-Core macht starken Gebrauch vom Cargo-Packagemanager und der offiziellen Package-Registry \textit{crates.io}.

\section{Konfiguration}
Die gesamte Konfiguration von Shulker-Core befindet sich in einer optionalen Textdatei. Da für jede mögliche Einstellung ein
Standartwert einprogrammiert ist, muss man nur nötige Änderungen vornehmen. Man muss sich also keine unnötigen Gedanken machen,
wie man jeden Wert in der Konfigurationsdatei definieren möchte.

Die Konfigurationsdatei hat das Dateiformat TOML und den Namen \textbf{config.toml}. Das Programm sucht im lokalen Ordner
nach dieser Datei. Das bedeutet, dass die config.toml-Datei sich im gleichen Ordner befinden muss, wie die Binary. Findet
Shulker-Core diese Datei nicht, werden alle Standartwerte übernommen. Als Warnung wird eine 
Warnmeldung an den Standart-Error ausgegeben.

\subsection{Mögliche Konfigurationen}
Es folgt eine Auflistung aller Konfigurationsschlüssel und ihrer Standartwerte:

\lstinline{master_password = "master123"}

Das Master-Passwort dient zur Identifikation der Admins. Das standartmäßige Adminpasswort sollte dringend durch 
ein Passwort vernünftiger Komplexität ersetzt werden. Es ist das einzige Passwort, das nicht gehashed wird. Dafür
gibt es drei Gründe: 
\begin{enumerate}
    \item Wer Zugriff zum Raspberry Pi hat, hat Zugriff zum Schloss.
    \item Dem Admin wird die Fähigkeit, ein einzigartiges Passwort zu erstellen, vorrausgesetzt.
    \item Bei Vergessen des Passworts kann der Admin es einfach wieder herausfinden. 
\end{enumerate}

\lstinline{send_socket_path = "/tmp/toShulkerServer.sock"}

Gibt den Pfad der Unix-Socket an, die zur Kommunikation mit Shulker-Connect dient; In Richtung Shulker-Core zu Shulker-Connect.
Um das System möglichst flexibel zu gestalten, wurde diese Konfiguration erschaffen. Man kann so leicht das System anpassen.

\lstinline{receive_socket_path = "/tmp/toShulkerCore.sock"}

Gibt den Pfad der Unix-Socket an, die zur Kommunikation mit Shulker-Connect dient; In Richtung Shulker-Connect zu Shulker-Core.
Um das System möglichst flexibel zu gestalten, wurde diese Konfiguration erschaffen. Man kann so leicht das System anpassen.

\lstinline{gpio_pin = 27}

Gibt an, welcher GPIO-Pin des Raspberries für die Ansteuerung des elektronischen Schlosses verwendet werden soll. Zu beachten:
soll das Schloss geschlossen sein, ist der GPIO-Pin auf High.

\lstinline{autolock_seconds = 24}

Die Dauer, die der GPIO-Pin bei richtiger Eingabe eines Pins (Passworts) auf High bleibt.

\lstinline{qr_code_link = "not setup!"}

Die Daten, die der QrCode in der M-Ansicht des Core-UIs enkodieren und zugänglich machen soll. Gedacht ist, dass der
Admin hier die IP des Raspberry Pis angibt.

\lstinline{hash_memory_size = 10}

Anzahl an Arbeitsspeicher, der für das Hashing der Pins verwendet wird. Genaueres wird später erklärt.

\lstinline{hash_iterations = 3}

Anzahl der Argon2id Durchläufe. Genaueres wird später erklärt.

\lstinline{hash_parallelism = 1}

Anzahl der Hash-Threads. Genaueres wird später erklärt.


\section{Abhängigkeiten}
Shulker-Core baut auf vielen \textit{crates}, also Codebibliotheken auf. Alle direkten Abhängigkeiten sind in der 
Cargo.toml-Datei nachvollziehbar. Es folgt eine kurze Auflistung und Erklärung der wichtigsten Abhängigkeiten:

\begin{itemize}
    \item \textbf{slint}; früher bekannt als SixtyFPS, ist ein User-Interface-Toolkit. Es ermöglicht relativ simples
    und plattformübergreifendes Programmieren von grafischen Benutzeroberflächen.

    \item \textbf{serde}; steht für \textbf{Ser}ialize und \textbf{De}serialize. Es ist das am bekannteste Framework,
    um Rust-Strukturen zu serialisieren und zu entserialisieren. Benutzt wird es, um die Kommunikation über Unix-Sockets zu
    ermöglichen.

    \item \textbf{argon2}; ist ein kryptographischer Passwort-Hashing Algorithmus. Das \textit{argon2}-crate implementiert diesen
    Algorithmus 100\%-ig mit Rust-Code. Die Variante Argon2id wird für das Unkenntlichmachen der Pins benutzt.

    \item \textbf{gpio-cdev}; wird verwendet, um den GPIO-Pin zu benutzen. Dabei wird auf die Linux-Kernel-API namens \textit{GPIO character device ABI}
    aufgebaut.

    \item \textbf{config}; ist ein crate, welches das Einlesen der Konfigurationsdatei erleichter.
    
    \item \textbf{chrono}; stellt Code zur Verfügung, um den Umgang mit Zeit und Datum zu erleichtern.
    
    \item \textbf{interprocess}; erleichtert die Verwendung von Unix-Sockets.
    
    \item \textbf{rustbreak}; eine einfache, in sich geschlossene Datenbank. Wird zur Speicherung von Daten verwendet.
    
    \item \textbf{qr\_code}; ermöglicht schnelles Erstellen von QrCodes. Wird verwendet um die Verbindung mit Shulker-Mobile zu
    erleichtern. Es ist das einzige crate, welches nicht von crates.io heruntergeladen wird. Das crate wird direkt von der Git-Repository auf
    Github kopiert.
\end{itemize}