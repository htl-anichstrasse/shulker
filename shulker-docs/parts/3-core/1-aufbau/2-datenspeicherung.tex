\chapter{Speicherung von Daten}
Shulker-Core speichert die vom Admin angelegten Pins ab. Alle Pins werden in einer Datei namens \textit{credentials} abgespeichert.
Es wird keine vollwertige Datenbank verwendet, um das System nicht zu groß zu gestalten und den Resourcenverbrauch auf dem 
Raspberry möglichst gering zu halten.

\section{Rustbreak}
Shulker-Core verwendet das crate \textit{rustbreak} zum Speichern von Daten. Rustbreak ist eine von der Ruby-Datenbank \textit{daybreak} inspirierte
Datenbank, welche Daten nur in einer Datei speichert. Es wird die Version 2.0 von rustbreak verwendet.

\section{Funktionsweise}
Shulker-Core lädt die gesamte \textit{credentials}-Datei beim Start in den Arbeitsspeicher. Wird ein eingegebener Pin
mit der Datenbank abgeglichen, muss so nur die geladene Datenbank abgefragt werden. Um Änderung an der Datenbank permament zu 
machen, wird die geladene Datei nach jeder Änderung wieder auf die Festplatte kopiert. Das System ist so konfiguriert, dass
ein Fehler beim Abspeichern der Datei keine Probleme verursacht. Die Ursprungsdatei bleibt in solchen Fällen erhalten. Es kommt 
zu keinem Datenverlust.

\section{Hashing von Pins}
Da die Nutzer von Shulker aus Einfachkeit möglicherweise Passwörter nutzen wollen, die sie an anderen Orten bereits verwenden, werden
alle Pins gehashed abgespeichert. Somit ist es dem Admin und einem möglichen Hacker nicht möglich, die originalen Pins nachzuvollziehen und
somit möglicherweise in anderen Bereichen Schaden anzurichten.

\subsection{Argon2}
Als Hashing-Algorithmus wird Argon2, genauer Argon2id verwendet. Argon2 wurde gewählt, da eine reine Rustimplementation des Algorithmus
existiert und der Algorithmus sich als Gewinner der \textit{Password Hashing Competition} behaupten.

\subsubsection{PHC-String}
Die unkenntlich gemachten Pins werden in der Form von PHC-Strings abgespeichert. Ein PHC-String besteht aus:
\begin{itemize}
    \item \textbf{id}: Kennzeichnung des verwendeten Hash-Algorithmus
    \item \textbf{version}: Version des Algorithmus
    \item \textbf{param}: mehrere Parameter, die dem Algorithmus gefüttert werden
    \item \textbf{salt}: zufällige Zeichenkette, welche das Schützen des Passworts weiter verbessert
    \item \textbf{hash}: der eigentliche Hash, also der unkenntlich gemachte Pin
\end{itemize}
\begin{figure}[H]
    \begin{center}
        \includegraphics[width=1\textwidth]{images/core/phc_string.png}
        \caption{Beispiel eines PHC-Strings}
    \end{center}
\end{figure}

\subsubsection{Konfiguration von Argon2id}
Shulker-Core nimmt drei Parameter für Argon2id an:

\begin{itemize}
    \item \textbf{Memory}: Die Menge an Arbeitsspeicher die der Algorithmus verwenden soll. Je höher dieser Wert ist,
    desto schwieriger ist es, den Hash zu berechnen. Somit wird es einem Angreifer ebenfalls schwerer, viele Hashes zu generieren
    und somit das originale Passwort herauszufinden. Wird in der Shulker-Core-Konfigurationsdatei als \textit{hash\_memory\_size} definiert.
    \item \textbf{Iterations}: Wie oft sich der Algorithmus durch die ausgewählte größe an Arbeitsspeicher durcharbeiten soll. Je höher, desto
    sicherer die Hashes. Wird in der Shulker-Core-Konfigurationsdatei als \textit{hash\_iterations} definiert.
    \item \textbf{Parallelism}: Wie viele Threads der Algorithmus verwenden soll. Desto höher dieser Wert, desto sicherer ist der Hash
    vor großen parallelen Attacken (zum Beispiel das Cracken mit Grafikkarten). Wird in der Shulker-Core-Konfigurationsdatei als \textit{hash\_parallelism} definiert.
\end{itemize}