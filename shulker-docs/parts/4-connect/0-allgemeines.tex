\lstset{language=[Sharp]C}
\chapter{Allgemeines}
\textit{Shulker-Connect} ist die Softwarelösung, um von externen Ressourcen aus mit dem Türschloss zu kommunizieren.

Dies haben wir mittels eines ASP.NET Core Web API Servers umgesetzt. Dieser Server stellt eine REST-API zur Verfügung, 
die von beliebigen Applikationen aus verwendet werden kann, um Shulker-Core Befehle mitzuteilen und Abfragen zu stellen.

In der Shulker-Zutrittsmanagement-Lösung wird der Shulker-Connect API-Server von Shulker-Mobile verwendet. 
Durch die Implementierung von Shulker-Connect könnten allerdings auch beliebige andere Applikationen, 
wie z.B.: eine Steuerungs-Website, das Türschloss steuern. 
So könnten später auf einfacher weise die Steuerungsmöglichkeiten des Schlosses erweitert werden.

\section{Verbindung zu Shulker-Core}
\textit{Shulker-Connect} muss mit \textit{Shulker-Core} kommunizieren, sodass eine API-Anfrage an Shulker-Connect tatsächlich
auch das Türschloss steuern kann.
Um eine stabile und effiziente Kommunikation mit der auch auf dem Raspberry PI laufenden Rust-Software Shulker-Core zu 
gewährleisten, haben wir dies mit \textit{POSIX-Sockets} implementiert.

Beim Start der Applikation werden zwei neue Threads erstellt: Ein Listener-Thread und ein Sender-Thread.
Beide Threads erstellen einen \textit{socket}, der auf eine Verbindung von \textit{Shulker-Core} wartet.
Wird diese hergestellt, sind die Threads bereit, Daten zu senden bzw. zu empfangen. 

\subsection{POSIX-Sockets}
\textit{POSIX local inter-process communication sockets} (auch Unix Domain Sockets oder IPC Sockets genannt) ermöglichen
eine bidirektionale Kommunikationsverbindung für die Interprozesskommunikation (IPC) auf UNIX basierenden Systemen.
Hierbei wird von beiden Kommunikations-Partnern eine Datei vereinbart, über diese die Kommunikation erfolgt. \cite{ipcsockets}
Die Kommunikation zwischen Shulker-Core und Shulker-Connect muss nicht verschlüsselt werden, da eine Kompromittierung des
Raspberry-Pi's die einzige Möglichkeit darstellt, diese Kommunikation mitzulesen. Eine Kompromittierung des Raspberry-Pi's
ist allerdings mehr als genug, um einem potentiellen Angreifer beliebige andere Möglichkeiten zu bieten, das Türschloss
beliebig zu manipulieren.

\section{Verschlüsselung der Anfragen von Shulker-Mobile zum API-Server}
Da es sich bei einem Haustürschloss um eine sehr wichtige und sichere Vorrichtung handelt, muss natürlich auch die
Kommunikation zu diesem auf sicherem Wege erfolgen. Das Türschloss soll schließlich mittels App von der ganzen Welt
aus steuerbar sein.

Um dieses Problem zu lösen und eine definitiv sichere und verschlüsselte Kommunikation zu gewährleisten, haben wir uns dazu 
entschieden, den Nutzer zu zwingen, einen VPN-Tunnel zum lokalen Netzwerk, in dem auch das Türschloss steht, herstellen
zu müssen, falls dieser sich nicht im lokalen Netzwerk befindet. So sicher wie das VPN-Protokoll, dass zur Verbindung genützt
wird ist, ist somit auch die Verbindung zum Türschloss.

Als zusätzlichen Vorteil bietet diese Lösung auch, dass das Türschloss nicht vom öffentlichen Internet aus erreichbar sein muss.
Dies sichert das Gesamtsystem noch einmal zusätzlich ab und schließt einen wichtigen Angriffsvektor. 