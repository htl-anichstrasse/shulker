\lstset{language=[Sharp]C}
\chapter{Allgemeines}
\textit{Shulker-Connect} ist die Softwarelösung, um von externen Ressourcen aus das Türschloss zu steuern.

Dies haben wir mittels eines ASP.NET Core Web API Servers umgesetzt. Dieser Server stellt eine REST-API zur Verfügung, 
auf die von beliebigen Applikationen aus zugegriffen werden kann, um Shulker-Core Befehle mitzuteilen und Abfragen zu stellen.

In der Shulker-Zutrittsmanagement-Lösung wird der Shulker-Connect API-Server von Shulker-Mobile verwendet. 
Durch die Implementierung von Shulker-Connect könnten allerdings auch beliebige andere Applikationen, 
wie z.B. eine Steuerungs-Website, das Türschloss steuern. 
So könnten später auf einfache Weise die Steuerungsmöglichkeiten des Schlosses erweitert werden.

\section{Authentifizierung am Server}
Immer dann, wenn Shulker-Mobile auf Routen von Shulker-Connect zugreifen möchte, muss sich der Client authentifizieren.
Diese Authentifizierung haben wir mittels eines Session-Systems umgesetzt.

Um dies am Server umzusetzen, haben wir eine \textit{SessionManager}-Klasse programmiert. Diese Klasse folgt dem
\textit{Singleton Pattern} und verwaltet alle aktiven Sessions. In Shulker bleibt eine Session 20 Minuten valide, dann
muss eine neue generiert werden.

Der Session-String selbst setzt sich aus 32 kryptografisch zufällig ausgewählten Buchstaben bzw. Zahlen zusammen.

\section{Verschlüsselung der Anfragen von Shulker-Mobile zum API-Server}
Da es sich bei einem Haustürschloss um eine sehr wichtige und sichere Vorrichtung handelt, muss natürlich auch die
Kommunikation zu diesem auf sicherem Wege erfolgen. Das Türschloss soll schließlich mittels App von der ganzen Welt
aus steuerbar sein.

Um dieses Problem zu lösen und eine definitiv sichere und verschlüsselte Kommunikation zu gewährleisten, haben wir uns dazu 
entschieden, den Nutzer zu zwingen, einen VPN-Tunnel zum lokalen Netzwerk, in dem auch das Türschloss steht, herstellen
zu müssen, falls dieser sich nicht im lokalen Netzwerk befindet. So sicher wie das VPN-Protokoll, das zur Verbindung genützt
wird ist, ist somit auch die Verbindung zum Türschloss.

Als zusätzlichen Vorteil bietet diese Lösung auch, dass das Türschloss nicht vom öffentlichen Internet aus erreichbar sein muss.
Dies sichert das Gesamtsystem noch einmal zusätzlich ab und schließt einen wichtigen Angriffsvektor.